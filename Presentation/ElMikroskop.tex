\documentclass[hyperref=unicode, presentation,10pt]{beamer}

\usepackage[absolute,overlay]{textpos}
\usepackage{array}
\usepackage{graphicx}
\usepackage{adjustbox}
\usepackage{mhchem}
\usepackage{chemfig}
\usepackage{caption}

%dělení slov
\usepackage{ragged2e}
\let\raggedright=\RaggedRight
%konec dělení slov

\addtobeamertemplate{frametitle}{
	\let\insertframetitle\insertsectionhead}{}
\addtobeamertemplate{frametitle}{
	\let\insertframesubtitle\insertsubsectionhead}{}

\makeatletter
\CheckCommand*\beamer@checkframetitle{\@ifnextchar\bgroup\beamer@inlineframetitle{}}
\renewcommand*\beamer@checkframetitle{\global\let\beamer@frametitle\relax\@ifnextchar\bgroup\beamer@inlineframetitle{}}
\makeatother
\setbeamercolor{section in toc}{fg=red}
\setbeamertemplate{section in toc shaded}[default][100]

\usepackage{fontspec}
\usepackage{unicode-math}

\usepackage{polyglossia}
\setdefaultlanguage{czech}

\def\uv#1{„#1“}

\mode<presentation>{\usetheme{default}}
\usecolortheme{crane}

\setbeamertemplate{footline}[frame number]

\title[Crisis]
{Elektronová mikroskopie a Brno}

\subtitle{BrNOC 12. 12. 2025}
\author{Zdeněk Moravec, hugo@chemi.muni.cz \\ \adjincludegraphics[height=45mm]{img/Aspergillus_niger_SEM.jpg}}
\date{}

\begin{document}
\begin{frame}
	\titlepage
\end{frame}

\section{Úvod}
\frame{
	\frametitle{}
	\vfill
	\begin{figure}
		\adjincludegraphics[width=1.1\textwidth]{img/Observable_Universe.png}
		\caption*{Pozorovatelný Vesmír.\footnote[frame]{Zdroj: \href{https://commons.wikimedia.org/wiki/File:Observable_Universe_Logarithmic_Map_(horizontal_layout_english_annotations).png}{Pablo Carlos Budassi/Commons}}}
	\end{figure}
	\vfill
}

\frame{
	\frametitle{}
	\vfill
	\begin{figure}
		\adjincludegraphics[width=.6\textwidth]{img/Intel_Celeron_D320_in_scanning_electron_microscope_8k.png}
		\caption*{SEM fotografie CPU.\footnote[frame]{Zdroj: \href{https://commons.wikimedia.org/wiki/File:Intel_Celeron_D320_in_scanning_electron_microscope_8k.png}{Piotr Krzemiński/Commons}}}
	\end{figure}
	\vfill
}

\section{Mikroskopie}
\frame{
	\frametitle{}
	\vfill
	\begin{columns}
		\begin{column}{.5\textwidth}
			\begin{figure}
				\adjincludegraphics[width=.9\textwidth]{img/Leitz_Mikroskop_1.jpg}
				\caption*{Světelný mikroskop.\footnote[frame]{Zdroj: \href{https://commons.wikimedia.org/wiki/File:Leitz_Mikroskop_1.jpg}{Holger.Ellgaard/Commons}}}
			\end{figure}
		\end{column}
		\begin{column}{.5\textwidth}
			\begin{figure}
				\adjincludegraphics[height=.7\textheight]{img/Schema_mikroskopu.png}
				\caption*{Princip funkce mikroskopu.\footnote[frame]{Zdroj: \href{https://commons.wikimedia.org/wiki/File:Schema_mikroskopu.svg}{Tlusťa/Commons}}}
			\end{figure}
		\end{column}
	\end{columns}
	\vfill
}

\frame{
	\frametitle{}
	\vfill
	\begin{figure}
		\adjincludegraphics[width=\textwidth]{img/EM_Spectrum_Properties.png}
		\caption*{Spektrum elektromagnetického záření.\footnote[frame]{Zdroj: \href{https://commons.wikimedia.org/wiki/File:EM_Spectrum_Properties_cz.svg}{Inductiveload/Commons}}}
	\end{figure}
	\vfill
}

\subsection{Historie mikroskopie}
\frame{
	\frametitle{}
	\vfill
	\textbf{Důležité milníky mikroskopie}\footnote[frame]{\href{https://en.wikipedia.org/wiki/Timeline_of_microscope_technology}{Timeline of microscope technology}}
	\begin{itemize}
		\item[1590] Hans Martens a Zacharias Janssen vytvořili první jednoduchý mikroskopy
		\item[1609] Galileo Galilei se využil obrácený teleskop k pozorování malých objektů
		\item[1873] Ernst Abbe formuluje teorii rozlišení – limit optické mikroskopie (\~200 nm)
		\item[1931] Ernst Ruska a Max Knoll konstruují první elektronový mikroskop (TEM)
		\item[1933] Ruska zlepšuje rozlišení EM pod hranici optických mikroskopů
		\item[1939] První komerční elektronový mikroskop (Siemens)
		\item[1981] První skenovací tunelový mikroskop (STM)
		\item[1981] První úspěšné využití cryo-EM.\footnote[frame]{\href{https://doi.org/10.1111/j.1365-2818.1981.tb02483.x}{Vitrification of pure water for electron microscopy}}
		\item[1986] První mikroskop atomárních sil (AFM)
	\end{itemize}
	\vfill
}

\subsection{Nobelovy ceny se vztahem k elektronové mikroskopii}
\frame{
	\frametitle{}
	\vfill
	\textbf{Nobelova cena za fyziku (1986), Ernst Ruska}\footnote[frame]{\href{https://www.nobelprize.org/prizes/physics/1986/summary/}{Nobel Prize in Physics 1986}}
	
	\uv{Za zásadní práci v oblasti elektronové optiky a za konstrukci prvního elektronového mikroskopu.}
	\vspace{2em}
	
	\textbf{Nobelova cena za chemii (2017), Jacques Dubochet, Joachim Frank, Richard Henderson}\footnote[frame]{\href{https://www.nobelprize.org/prizes/chemistry/2017/summary/}{Nobel Prize in Chemistry 2017}}
	
	\uv{Za vývoj kryo-elektronové mikroskopie pro vysokorozlišovací určování struktur biomolekul v roztoku.}
	\vspace{2em}
	
	\textbf{Nobelova cena za chemii (2024), David Baker}\footnote[frame]{\href{https://www.nobelprize.org/prizes/chemistry/2024/summary/}{Nobel Prize in Chemistry 2024}}
	
	\uv{Za návrh nových proteinů, které mimo jiné slouží jako značky pro pokročilou elektronovou mikroskopii.}
	\vfill
}

\section{Elektronová mikroskopie}

\frame{
	\frametitle{}
	\vfill
	\begin{itemize}
		\item Místo světla využívá proud urychlených elektronů, což umožňuje zlepšit rozlišení fotografie.
		\item Získané obrázky jsou černobílé, ale je možné je kolorovat na základě dalších dat, např. prvkového složení.
		\item Druhy elektronové mikroskopie:
		\begin{itemize}
			\item SEM -- skenovací elektronová mikroskopie
			\item TEM -- transmisní elektronová mikroskopie
			\item cryo-EM -- kryoelektronová mikroskopie
			\item AFM -- mikroskopie atomárních sil
		\end{itemize}
		\item EDX -- energiově dispezní RTG spektroskopie (Energy-dispersive \mbox{X-ray} spectroscopy). Analytická technika poskytující prvkové složení vzorku.
		\item Snímání objektu probíhá zpravidla ve vakuu a vzorky by měly být vodivé.
	\end{itemize}
	\vfill
}

\frame{
	\frametitle{}
	\vfill
	\begin{columns}
		\begin{column}{.7\textwidth}
			\begin{figure}
				\adjincludegraphics[width=\textwidth]{img/Schema_MEB.png}
				\caption*{Schéma SEM.\footnote[frame]{Zdroj: \href{https://commons.wikimedia.org/wiki/File:Schema_MEB_(en).svg}{Steff/Commons}}}
			\end{figure}
		\end{column}
		\begin{column}{.3\textwidth}
			\begin{figure}
				\adjincludegraphics[width=1.3\textwidth]{img/Electron_emission.png}
				\caption*{Emise elektronů.\footnote[frame]{Zdroj: \href{https://commons.wikimedia.org/wiki/File:Electron_emission_mechanisms.svg}{Rob Hurt/Commons}}}
			\end{figure}
		\end{column}
	\end{columns}
	\vfill
}

\frame{
	\frametitle{}
	\vfill
	\begin{figure}
		\adjincludegraphics[height=.75\textheight]{img/Gold_Spider_SEM_sample.jpg}
		\caption*{Pozlacený pavouk, pro pozorování pomocí SEM.\footnote[frame]{Zdroj: \href{https://commons.wikimedia.org/wiki/File:Gold_Spider_SEM_sample.jpg}{Toby Hudson/Commons}}}
	\end{figure}
	\vfill
}

\frame{
	\frametitle{}
	\vfill
	\begin{figure}
		\adjincludegraphics[height=.67\textheight]{img/Germanium_Telluride_nanowires.jpg}
		\caption*{Nanovlákna GeTe.\footnote[frame]{Zdroj: \href{https://commons.wikimedia.org/wiki/File:Germanium_Telluride_nanowires.jpg}{Fionán/Commons}}}
	\end{figure}
	\vfill
}

\frame{
	\frametitle{}
	\vfill
	\begin{figure}
		\adjincludegraphics[height=.67\textheight]{img/Quasimodopsis_riedeli_SEM.jpg}
		\caption*{SEM snímek \textit{Quasimodopsis riedeli}.\footnote[frame]{Zdroj: \href{https://commons.wikimedia.org/wiki/File:Quasimodopsis_riedeli_SEM.jpg}{Michael S. Caterino/Commons}}}
	\end{figure}
	\vfill
}

\frame{
	\frametitle{}
	\vfill
	\begin{figure}
		\adjincludegraphics[height=.67\textheight]{img/EDS_-_Rimicaris_exoculata.png}
		\caption*{Ukázka EDX spektra.\footnote[frame]{Zdroj: \href{https://commons.wikimedia.org/wiki/File:EDS_-_Rimicaris_exoculata.png}{Hat'nCoat/Commons}}}
	\end{figure}
	\vfill
}

\frame{
	\frametitle{}
	\vfill
	\begin{columns}
		\begin{column}{.7\textwidth}
			\begin{figure}
				\adjincludegraphics[height=.63\textheight]{img/TEM_philips_EM_430.jpg}
				\caption*{TEM mikroskop.\footnote[frame]{Zdroj: \href{https://commons.wikimedia.org/wiki/File:TEM_philips_EM_430.jpg}{KristianMolhave/Commons}}}
			\end{figure}
		\end{column}
		\begin{column}{.3\textwidth}
			\begin{figure}
				\adjincludegraphics[height=.63\textheight]{img/Electron_microscope.png}
				\caption*{Schéma TEM mikroskopu.\footnote[frame]{Zdroj: \href{https://commons.wikimedia.org/wiki/File:Electron_microscope.svg}{Superborsuk/Commons}}}
			\end{figure}
		\end{column}
	\end{columns}
	\vfill
}

\frame{
	\frametitle{}
	\vfill
	\begin{figure}
		\adjincludegraphics[height=.65\textheight]{img/TEM_images_of_silica.jpg}
		\caption*{TEM snímek hybridní siliky.\footnote[frame]{Zdroj: \href{https://commons.wikimedia.org/wiki/File:TEM_images_of_silica-organic_matrix-based_suspension.jpg}{Florentyna/Commons}}}
	\end{figure}
	\vfill
}

\frame{
	\frametitle{}
	\vfill
	\begin{figure}
		\adjincludegraphics[height=.67\textheight]{img/AFMsetup.jpg}
		\caption*{Schéma AFM.\footnote[frame]{Zdroj: \href{https://commons.wikimedia.org/wiki/File:AFMsetup.jpg}{KristianMolhave/Commons}}}
	\end{figure}
	\vfill
}

\frame{
	\frametitle{}
	\vfill
	\begin{figure}
		\adjincludegraphics[height=.67\textheight]{img/AFM_-_detail.jpg}
		\caption*{AFM mikroskop.\footnote[frame]{Zdroj: \href{https://commons.wikimedia.org/wiki/File:AFM_-_detail.jpg}{Laundry/Commons}}}
	\end{figure}
	\vfill
}

\frame{
	\frametitle{}
	\vfill
	\begin{figure}
		\adjincludegraphics[height=.67\textheight]{img/AFM_3D_Topography.jpg}
		\caption*{AFM snímek nanočástic palladia.\footnote[frame]{Zdroj: \href{https://commons.wikimedia.org/wiki/File:AFM_3D_Topography_Image_of_Palladium_Nanoparticles_on_Chitosan_Film.tif}{Mehrabanian/Commons}}}
	\end{figure}
	\vfill
}

\section{Elektronová mikroskopie a Brno}
\frame{
	\frametitle{}
	\vfill
	\begin{columns}
		\begin{column}{.5\textwidth}
			\begin{figure}
				\adjincludegraphics[height=.7\textheight]{img/Brno_Montage_IV.png}
				\caption*{Brno.\footnote[frame]{Zdroj: \href{https://commons.wikimedia.org/wiki/File:Brno_Montage_IV.png}{Commons}}}
			\end{figure}
		\end{column}
		\begin{column}{.5\textwidth}
			\begin{figure}
				\adjincludegraphics[height=.7\textheight]{img/Titan_Krios_University_of_Leeds.jpg}
				\caption*{Titan Krios Cryo-EM.\footnote[frame]{Zdroj: \href{https://commons.wikimedia.org/wiki/File:Titan_Krios_University_of_Leeds.jpg}{Hiramano92/Commons}}}
			\end{figure}
		\end{column}
	\end{columns}
	\vfill
}

\subsection{Armin Delong}

\end{document}