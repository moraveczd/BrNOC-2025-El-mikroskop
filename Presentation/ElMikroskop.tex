\documentclass[hyperref=unicode, presentation,10pt]{beamer}

\usepackage[absolute,overlay]{textpos}
\usepackage{array}
\usepackage{graphicx}
\usepackage{adjustbox}
\usepackage{mhchem}
\usepackage{chemfig}
\usepackage{caption}

%dělení slov
\usepackage{ragged2e}
\let\raggedright=\RaggedRight
%konec dělení slov

\addtobeamertemplate{frametitle}{
	\let\insertframetitle\insertsectionhead}{}
\addtobeamertemplate{frametitle}{
	\let\insertframesubtitle\insertsubsectionhead}{}

\makeatletter
\CheckCommand*\beamer@checkframetitle{\@ifnextchar\bgroup\beamer@inlineframetitle{}}
\renewcommand*\beamer@checkframetitle{\global\let\beamer@frametitle\relax\@ifnextchar\bgroup\beamer@inlineframetitle{}}
\makeatother
\setbeamercolor{section in toc}{fg=red}
\setbeamertemplate{section in toc shaded}[default][100]

\usepackage{fontspec}
\usepackage{unicode-math}

\usepackage{polyglossia}
\setdefaultlanguage{czech}

\def\uv#1{„#1“}

\mode<presentation>{\usetheme{default}}
\usecolortheme{crane}

\setbeamertemplate{footline}[frame number]

\title[Crisis]
{Elektronová mikroskopie a Brno}

\subtitle{BrNOC 5. 12. 2025}
\author{Zdeněk Moravec, hugo@chemi.muni.cz \\ \adjincludegraphics[height=45mm]{img/Aspergillus_niger_SEM.jpg}}
\date{}

\begin{document}
\begin{frame}
	\titlepage
\end{frame}

\section{Úvod}

\frame{
	\frametitle{}
	\vfill
	\begin{itemize}
		\item Pro analýzu prášků a povrchových vrstev se využívá \textit{elektronová mikroskopie}.
		\item Místo světla využívá proud urychlených elektronů, což umožňuje zlepšit rozlišení fotografie.
		\item Získané obrázky jsou černobílé, ale je možné je kolorovat na základě dalších dat, např. prvkového složení.
		\item Druhy elektronové mikroskopie:
		\begin{itemize}
			\item SEM -- skenovací elektronová mikroskopie
			\item TEM -- transmisní elektronová mikroskopie
			\item AFM -- mikroskopie atomárních sil
		\end{itemize}
		\item EDX -- energiově dispezní RTG spektroskopie (Energy-dispersive \mbox{X-ray} spectroscopy). Analytická technika poskytující prvkové složení vzorku.
		\item Snímání objektu probíhá zpravidla ve vakuu a vzorky by měly být vodivé.
	\end{itemize}
	\vfill
}

\frame{
	\frametitle{}
	\vfill
	\begin{columns}
		\begin{column}{.7\textwidth}
			\begin{figure}
				\adjincludegraphics[width=\textwidth]{img/Schema_MEB.png}
				\caption*{Schéma SEM.\footnote[frame]{Zdroj: \href{https://commons.wikimedia.org/wiki/File:Schema_MEB_(en).svg}{Steff/Commons}}}
			\end{figure}
		\end{column}
		\begin{column}{.3\textwidth}
			\begin{figure}
				\adjincludegraphics[width=1.3\textwidth]{img/Electron_emission.png}
				\caption*{Emise elektronů.\footnote[frame]{Zdroj: \href{https://commons.wikimedia.org/wiki/File:Electron_emission_mechanisms.svg}{Rob Hurt/Commons}}}
			\end{figure}
		\end{column}
	\end{columns}
	\vfill
}

\frame{
	\frametitle{}
	\vfill
	\begin{figure}
		\adjincludegraphics[height=.67\textheight]{img/Germanium_Telluride_nanowires.jpg}
		\caption*{Nanovlákna GeTe.\footnote[frame]{Zdroj: \href{https://commons.wikimedia.org/wiki/File:Germanium_Telluride_nanowires.jpg}{Fionán/Commons}}}
	\end{figure}
	\vfill
}

\frame{
	\frametitle{}
	\vfill
	\begin{figure}
		\adjincludegraphics[height=.67\textheight]{img/Quasimodopsis_riedeli_SEM.jpg}
		\caption*{SEM snímek \textit{Quasimodopsis riedeli}.\footnote[frame]{Zdroj: \href{https://commons.wikimedia.org/wiki/File:Quasimodopsis_riedeli_SEM.jpg}{Michael S. Caterino/Commons}}}
	\end{figure}
	\vfill
}

\frame{
	\frametitle{}
	\vfill
	\begin{figure}
		\adjincludegraphics[height=.67\textheight]{img/EDS_-_Rimicaris_exoculata.png}
		\caption*{Ukázka EDX spektra.\footnote[frame]{Zdroj: \href{https://commons.wikimedia.org/wiki/File:EDS_-_Rimicaris_exoculata.png}{Hat'nCoat/Commons}}}
	\end{figure}
	\vfill
}

\frame{
	\frametitle{}
	\vfill
	\begin{columns}
		\begin{column}{.7\textwidth}
			\begin{figure}
				\adjincludegraphics[height=.63\textheight]{img/TEM_philips_EM_430.jpg}
				\caption*{TEM mikroskop.\footnote[frame]{Zdroj: \href{https://commons.wikimedia.org/wiki/File:TEM_philips_EM_430.jpg}{KristianMolhave/Commons}}}
			\end{figure}
		\end{column}
		\begin{column}{.3\textwidth}
			\begin{figure}
				\adjincludegraphics[height=.63\textheight]{img/Electron_microscope.png}
				\caption*{Schéma TEM mikroskopu.\footnote[frame]{Zdroj: \href{https://commons.wikimedia.org/wiki/File:Electron_microscope.svg}{Superborsuk/Commons}}}
			\end{figure}
		\end{column}
	\end{columns}
	\vfill
}

\frame{
	\frametitle{}
	\vfill
	\begin{figure}
		\adjincludegraphics[height=.65\textheight]{img/TEM_images_of_silica.jpg}
		\caption*{TEM snímek hybridní siliky.\footnote[frame]{Zdroj: \href{https://commons.wikimedia.org/wiki/File:TEM_images_of_silica-organic_matrix-based_suspension.jpg}{Florentyna/Commons}}}
	\end{figure}
	\vfill
}

\frame{
	\frametitle{}
	\vfill
	\begin{figure}
		\adjincludegraphics[height=.67\textheight]{img/AFMsetup.jpg}
		\caption*{Schéma AFM.\footnote[frame]{Zdroj: \href{https://commons.wikimedia.org/wiki/File:AFMsetup.jpg}{KristianMolhave/Commons}}}
	\end{figure}
	\vfill
}

\frame{
	\frametitle{}
	\vfill
	\begin{figure}
		\adjincludegraphics[height=.67\textheight]{img/AFM_-_detail.jpg}
		\caption*{AFM mikroskop.\footnote[frame]{Zdroj: \href{https://commons.wikimedia.org/wiki/File:AFM_-_detail.jpg}{Laundry/Commons}}}
	\end{figure}
	\vfill
}

\frame{
	\frametitle{}
	\vfill
	\begin{figure}
		\adjincludegraphics[height=.67\textheight]{img/AFM_3D_Topography.jpg}
		\caption*{AFM snímek nanočástic palladia.\footnote[frame]{Zdroj: \href{https://commons.wikimedia.org/wiki/File:AFM_3D_Topography_Image_of_Palladium_Nanoparticles_on_Chitosan_Film.tif}{Mehrabanian/Commons}}}
	\end{figure}
	\vfill
}

\end{document}