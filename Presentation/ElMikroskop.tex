\documentclass[hyperref=unicode, presentation,10pt]{beamer}

\usepackage[absolute,overlay]{textpos}
\usepackage{array}
\usepackage{graphicx}
\usepackage{adjustbox}
\usepackage{mhchem}
\usepackage{chemfig}
\usepackage{caption}

%dělení slov
\usepackage{ragged2e}
\let\raggedright=\RaggedRight
%konec dělení slov

\addtobeamertemplate{frametitle}{
	\let\insertframetitle\insertsectionhead}{}
\addtobeamertemplate{frametitle}{
	\let\insertframesubtitle\insertsubsectionhead}{}

\makeatletter
\CheckCommand*\beamer@checkframetitle{\@ifnextchar\bgroup\beamer@inlineframetitle{}}
\renewcommand*\beamer@checkframetitle{\global\let\beamer@frametitle\relax\@ifnextchar\bgroup\beamer@inlineframetitle{}}
\makeatother
\setbeamercolor{section in toc}{fg=red}
\setbeamertemplate{section in toc shaded}[default][100]

\usepackage{fontspec}
\usepackage{unicode-math}

\usepackage{polyglossia}
\setdefaultlanguage{czech}

\def\uv#1{„#1“}

\mode<presentation>{\usetheme{default}}
\usecolortheme{crane}

\setbeamertemplate{footline}[frame number]

\title[Crisis]
{Elektronová mikroskopie a Brno}

\subtitle{BrNOC 12. 12. 2025}
\author{Zdeněk Moravec, hugo@chemi.muni.cz \\ \adjincludegraphics[height=45mm]{img/Aspergillus_niger_SEM.jpg}}
\date{}

\begin{document}
\begin{frame}
	\titlepage
\end{frame}

\section{Úvod}
\frame{
	\frametitle{}
	\vfill
	\begin{figure}
		\adjincludegraphics[width=1.1\textwidth]{img/Observable_Universe.png}
		\caption*{Pozorovatelný Vesmír.\footnote[frame]{Zdroj: \href{https://commons.wikimedia.org/wiki/File:Observable_Universe_Logarithmic_Map_(horizontal_layout_english_annotations).png}{Pablo Carlos Budassi/Commons}}}
	\end{figure}
	\vfill
}

\frame{
	\frametitle{}
	\vfill
	\begin{figure}
		\adjincludegraphics[width=.6\textwidth]{img/Intel_Celeron_D320_in_scanning_electron_microscope_8k.png}
		\caption*{SEM fotografie CPU.\footnote[frame]{Zdroj: \href{https://commons.wikimedia.org/wiki/File:Intel_Celeron_D320_in_scanning_electron_microscope_8k.png}{Piotr Krzemiński/Commons}}}
	\end{figure}
	\vfill
}

\section{Mikroskopie}
\frame{
	\frametitle{}
	\vfill
	\begin{columns}
		\begin{column}{.5\textwidth}
			\begin{figure}
				\adjincludegraphics[width=.9\textwidth]{img/Leitz_Mikroskop_1.jpg}
				\caption*{Světelný mikroskop.\footnote[frame]{Zdroj: \href{https://commons.wikimedia.org/wiki/File:Leitz_Mikroskop_1.jpg}{Holger.Ellgaard/Commons}}}
			\end{figure}
		\end{column}
		\begin{column}{.5\textwidth}
			\begin{figure}
				\adjincludegraphics[height=.7\textheight]{img/Schema_mikroskopu.png}
				\caption*{Princip funkce mikroskopu.\footnote[frame]{Zdroj: \href{https://commons.wikimedia.org/wiki/File:Schema_mikroskopu.svg}{Tlusťa/Commons}}}
			\end{figure}
		\end{column}
	\end{columns}
	\vfill
}

\frame{
	\frametitle{}
	\vfill
	\begin{figure}
		\adjincludegraphics[width=\textwidth]{img/EM_Spectrum_Properties.png}
		\caption*{Spektrum elektromagnetického záření.\footnote[frame]{Zdroj: \href{https://commons.wikimedia.org/wiki/File:EM_Spectrum_Properties_cz.svg}{Inductiveload/Commons}}}
	\end{figure}
	\vfill
}

\subsection{Historie mikroskopie}
\frame{
	\frametitle{}
	\vfill
	\textbf{Důležité milníky mikroskopie}\footnote[frame]{\href{https://en.wikipedia.org/wiki/Timeline_of_microscope_technology}{Timeline of microscope technology}}
	\begin{itemize}
		\item[1590] Hans Martens a Zacharias Janssen vytvořili první jednoduchý mikroskop
		\item[1609] Galileo Galilei se využil obrácený teleskop k pozorování malých objektů
		\item[1873] Ernst Abbe formuluje teorii rozlišení – limit optické mikroskopie (\~200 nm)
		\item[1931] Ernst Ruska a Max Knoll konstruují první elektronový mikroskop (TEM)
		\item[1933] Ruska zlepšuje rozlišení EM pod hranici optických mikroskopů
		\item[1939] První komerční elektronový mikroskop (Siemens)
		\item[1981] První skenovací tunelový mikroskop (STM)
		\item[1981] První úspěšné využití cryo-EM.\footnote[frame]{\href{https://doi.org/10.1111/j.1365-2818.1981.tb02483.x}{Vitrification of pure water for electron microscopy}}
		\item[1986] První mikroskop atomárních sil (AFM)
	\end{itemize}
	\vfill
}

\subsection{Nobelovy ceny se vztahem k elektronové mikroskopii}
\frame{
	\frametitle{}
	\vfill
	\textbf{Nobelova cena za fyziku (1986), Ernst Ruska}\footnote[frame]{\href{https://www.nobelprize.org/prizes/physics/1986/summary/}{Nobel Prize in Physics 1986}}

	\uv{Za zásadní práci v oblasti elektronové optiky a za konstrukci prvního elektronového mikroskopu.}
	\vspace{2em}

	\textbf{Nobelova cena za chemii (2017), Jacques Dubochet, Joachim Frank, Richard Henderson}\footnote[frame]{\href{https://www.nobelprize.org/prizes/chemistry/2017/summary/}{Nobel Prize in Chemistry 2017}}

	\uv{Za vývoj kryo-elektronové mikroskopie pro vysokorozlišovací určování struktur biomolekul v roztoku.}
	\vspace{2em}

	\textbf{Nobelova cena za chemii (2024), David Baker}\footnote[frame]{\href{https://www.nobelprize.org/prizes/chemistry/2024/summary/}{Nobel Prize in Chemistry 2024}}

	\uv{Za návrh nových proteinů, které mimo jiné slouží jako značky pro pokročilou elektronovou mikroskopii.}
	\vfill
}

\section{Elektronová mikroskopie}
\frame{
	\frametitle{}
	\vfill
	\textbf{Elektronová mikroskopie}
	\begin{itemize}
		\item Místo světla využívá proud urychlených elektronů, což umožňuje zlepšit rozlišení fotografie.
		\item Umožňují zvětšení obrazu až 2 000 000$\times$, to umožňuje rozlišení až na atomární úrovni (50~pm = 5.10$^{-11}$ m).
		\item Získané obrázky jsou černobílé, ale je možné je kolorovat na základě dalších dat, např. prvkového složení.
		\item Druhy elektronové mikroskopie:
		\begin{itemize}
			\item SEM -- skenovací elektronová mikroskopie
			\item TEM -- transmisní elektronová mikroskopie
			\item cryo-EM -- kryoelektronová mikroskopie
			\item AFM -- mikroskopie atomárních sil
		\end{itemize}
		\item EDX -- energiově dispezní RTG spektroskopie (Energy-dispersive \mbox{X-ray} spectroscopy). Analytická technika poskytující prvkové složení vzorku.
		\item Snímání objektu probíhá zpravidla ve vakuu.
	\end{itemize}
	\vfill
}

\frame{
	\frametitle{}
	\vfill
	\begin{figure}
		\adjincludegraphics[height=.75\textheight]{img/MicroscopesOverview.png}
		\caption*{Typy mikroskopů.\footnote[frame]{Zdroj: \href{https://commons.wikimedia.org/wiki/File:MicroscopesOverview.svg}{FDominec/Commons}}}
	\end{figure}
	\vfill
}

\subsection{Elektron}
\frame{
	\frametitle{}
	\vfill
	\begin{columns}
		\begin{column}{.65\textwidth}
			\begin{itemize}
				\item Atom -- skládá se z elektronového obalu a atomového jádra
				\item Elektronový obal -- tvoří většinu objemu atomu, ale je skoro prázdný
				\item Atomové jádro -- malý objem, ale obsahuje většinu hmoty atomu
				\item Periodická tabulka prvků -- atomy (prvky) seřazené podle hmotnosti (počtu protonů)
			\end{itemize}
		\end{column}
		\begin{column}{.4\textwidth}
			\begin{figure}
				\adjincludegraphics[width=\textwidth]{img/Sodium_Atom.png}
				\caption*{Model atomu sodíku.\footnote[frame]{Zdroj: \href{https://commons.wikimedia.org/wiki/File:Sodium_Atom.png}{Plazmi/Commons}}}
			\end{figure}
		\end{column}
	\end{columns}
	\vfill
}

\frame{
	\frametitle{}
	\vfill
	\begin{columns}
		\begin{column}{.6\textwidth}
			\textbf{Elektron}
			\begin{itemize}
				\item Elektron je subatomární částice se záporným elektrickým nábojem.
				\item Byl objeven roku 1887 anglickým fyzikem J. J. Thompsonem.
				\item Patří mezi fermiony, platí pro něj \textit{Pauliho vylučovací princip}.
				\item Tvoří elektronový obal atomu, kde jsou elektrony uspořádány do vrstev a pohybují se v atomových orbitalech.
			\end{itemize}
		\end{column}
		\begin{column}{.4\textwidth}
			\begin{figure}
				\adjincludegraphics[width=\textwidth]{img/Cyclotron_motion_wider_view.jpg}
				\caption*{Proud elektronů usměrněný do kruhového pohybu.\footnote[frame]{Zdroj: \href{https://commons.wikimedia.org/wiki/File:Cyclotron_motion_wider_view.jpg}{Marcin Białek/Commons}}}
			\end{figure}
		\end{column}
	\end{columns}
	\vfill
}

\frame{
	\frametitle{}
	\vfill
	\begin{itemize}
		\item \textbf{Elektronové dělo} je zdroj úzkého proudu elektronů.
		\item Pracuje na principu vytrhávání elektronů působením silného elektrického pole nebo emise vyvolané teplotou, plazmatem, $\ldots$
	\end{itemize}
	\begin{columns}
		\begin{column}{.5\textwidth}
			\begin{figure}
				\adjincludegraphics[width=\textwidth]{img/31LK3b_electron_gun.jpg}
				\caption*{Elektronové dělo z CRT obrazovky.\footnote[frame]{Zdroj: \href{https://commons.wikimedia.org/wiki/File:31LK3b_electron_gun.jpg}{Mister rf/Commons}}}
			\end{figure}
		\end{column}
		\begin{column}{.5\textwidth}
			\begin{figure}
				\adjincludegraphics[width=\textwidth]{img/Schottky-Emitter_01.jpg}
				\caption*{Schottkyho katoda.\footnote[frame]{Zdroj: \href{https://commons.wikimedia.org/wiki/File:Schottky-Emitter_01.jpg}{ErwinMeier/Commons}}}
			\end{figure}
		\end{column}
	\end{columns}
	\vfill
}

\frame{
	\frametitle{}
	\vfill
	\begin{columns}
		\begin{column}{.65\textwidth}
			\begin{itemize}
				\item V elektronových mikroskopech se využívá krystal wolframu s velmi tenkým hrotem, který je vystaven velmi silnému elektrickému poli.
				\item Uvolněné elektrony jsou soustavou nabitých destiček urychleny až na cca 70 \% rychlosti světla.
				\item Vše probíhá ve velmi vysokém vakuu.
			\end{itemize}
		\end{column}
		\begin{column}{.4\textwidth}
			\begin{figure}
				\adjincludegraphics[width=\textwidth]{img/Wolfram.jpg}
				\caption*{Wolfram.\footnote[frame]{Zdroj: \href{https://commons.wikimedia.org/wiki/File:Wolfram_evaporated_crystals_and_1cm3_cube.jpg}{Alchemist-hp/Commons}}}
			\end{figure}
		\end{column}
	\end{columns}
	\vfill
}

\subsection{Transmisní elektronová mikroskopie}
\frame{
	\frametitle{}
	\vfill
	\begin{itemize}
		\item \textbf{Transmisní elektronová mikroskopie} (TEM) je založena na průchodu elektronového svazku celým vzorkem.
		\item Vysokoenergetické elektrony (60--300 keV) jsou fokusovány do úzkého svazku pomocí elektromagnetických čoček.
		\item Vzorek musí být velmi tenký (desítky až stovky nanometrů), aby elektrony mohly projít. Při průchodu dochází k rozptylu elektronů podle struktury a hustoty materiálu.
		\item Elektrony, které prošly vzorkem, jsou zachyceny na fluorescenčním stínítku nebo detektoru. Rozdíly v intenzitě (způsobené rozptylem) vytvářejí obraz s vysokým rozlišením.
		\item TEM dosahuje rozlišení až na úroveň atomů (řádově 0,1 nm), což je mnohem vyšší než u SEM.
		\item Umožňuje studovat vnitřní strukturu vzorku (ne jen povrch).
		\item Vhodné pro krystalografii, nanostruktury, biologické vzorky (po speciální přípravě).
	\end{itemize}
	\vfill
}

\frame{
	\frametitle{}
	\vfill
	\begin{columns}
		\begin{column}{.7\textwidth}
			\begin{figure}
				\adjincludegraphics[height=.63\textheight]{img/TEM_philips_EM_430.jpg}
				\caption*{TEM mikroskop.\footnote[frame]{Zdroj: \href{https://commons.wikimedia.org/wiki/File:TEM_philips_EM_430.jpg}{KristianMolhave/Commons}}}
			\end{figure}
		\end{column}
		\begin{column}{.3\textwidth}
			\begin{figure}
				\adjincludegraphics[height=.63\textheight]{img/Electron_microscope.png}
				\caption*{Schéma TEM mikroskopu.\footnote[frame]{Zdroj: \href{https://commons.wikimedia.org/wiki/File:Electron_microscope.svg}{Superborsuk/Commons}}}
			\end{figure}
		\end{column}
	\end{columns}
	\vfill
}

\frame{
	\frametitle{}
	\vfill
	\begin{figure}
		\adjincludegraphics[height=.65\textheight]{img/TEM_images_of_silica.jpg}
		\caption*{TEM snímek hybridní siliky.\footnote[frame]{Zdroj: \href{https://commons.wikimedia.org/wiki/File:TEM_images_of_silica-organic_matrix-based_suspension.jpg}{Florentyna/Commons}}}
	\end{figure}
	\vfill
}

\subsection{Skenovací elektronová mikroskopie}
\frame{
	\frametitle{}
	\vfill
	\begin{itemize}
		\item \textbf{Skenovací elektronová mikroskopie} (SEM) je založena na skenování povrchu vzorku elektronovým svazkem a detekuje signály vznikající při interakci elektronů s materiálem.
		\item Vysokoenergetické elektrony (obvykle 1–30 keV) jsou fokusovány do úzkého svazku pomocí elektromagnetických čoček.
		\item Svazek postupně „projíždí“ po povrchu vzorku v rastru.
		\item Vzorek musí být vodivý. Nevodivé vzorky je nutné pokovit nebo pokrýt velmi tenkou vrstvou grafitu, jinak dochází k jejich nabíjení.
		\item Elektrony pronikají do povrchové vrstvy vzorku a vyvolávají různé signály:
		\begin{itemize}
			\item Sekundární elektrony -- poskytují detailní informace o topografii povrchu.
			\item Odražené (zpětně rozptýlené) elektrony -- poskytují informace o složení a kontrastu.
			\item Rentgenové záření -- chemická analýza (EDS -- Energiově Disperzní Spektroskopie).
		\end{itemize}
		\item Detektory zachytí tyto signály a počítač je převede na obraz s vysokým rozlišením (typicky nanometry).
	\end{itemize}
	\vfill
}

\frame{
	\frametitle{}
	\vfill
	\begin{figure}
		\adjincludegraphics[height=.7\textheight]{img/Gold_Spider_SEM_sample.jpg}
		\caption*{Pozlacený pavouk, pro pozorování pomocí SEM.\footnote[frame]{Zdroj: \href{https://commons.wikimedia.org/wiki/File:Gold_Spider_SEM_sample.jpg}{Toby Hudson/Commons}}}
	\end{figure}
	\vfill
}

\frame{
	\frametitle{}
	\vfill
	\begin{columns}
		\begin{column}{.7\textwidth}
			\begin{figure}
				\adjincludegraphics[width=\textwidth]{img/Schema_MEB.png}
				\caption*{Schéma SEM.\footnote[frame]{Zdroj: \href{https://commons.wikimedia.org/wiki/File:Schema_MEB_(en).svg}{Steff/Commons}}}
			\end{figure}
		\end{column}
		\begin{column}{.3\textwidth}
			\begin{figure}
				\adjincludegraphics[width=1.3\textwidth]{img/Electron_emission.png}
				\caption*{Emise elektronů.\footnote[frame]{Zdroj: \href{https://commons.wikimedia.org/wiki/File:Electron_emission_mechanisms.svg}{Rob Hurt/Commons}}}
			\end{figure}
		\end{column}
	\end{columns}
	\vfill
}

\frame{
	\frametitle{}
	\vfill
	\begin{figure}
		\adjincludegraphics[height=.67\textheight]{img/Granat_zoniert.jpg}
		\caption*{SEM snímek granátu v BSE režimu.\footnote[frame]{Zdroj: \href{https://commons.wikimedia.org/wiki/File:Granat_zoniert.jpg}{Egon Bernabè/Commons}}}
	\end{figure}
	\vfill
}

\frame{
	\frametitle{}
	\vfill
	\begin{figure}
		\adjincludegraphics[height=.67\textheight]{img/EDS_-_Rimicaris_exoculata.png}
		\caption*{Ukázka EDX spektra.\footnote[frame]{Zdroj: \href{https://commons.wikimedia.org/wiki/File:EDS_-_Rimicaris_exoculata.png}{Hat'nCoat/Commons}}}
	\end{figure}
	\vfill
}

\frame{
	\frametitle{}
	\vfill
	\textbf{Využití SEM}
	\begin{itemize}
		\item Materiálové vědy
		\begin{itemize}
			\item Studium mikrostruktury kovů, keramiky, kompozitů.
			\item Analýza poréznosti, trhlin, povrchových úprav.
		\end{itemize}
		\item Biologie a medicína
		\begin{itemize}
			\item Zobrazení povrchu buněk, tkání, mikroorganismů.
			\item Studium morfologie biomateriálů (implantáty, kostní náhrady).
		\end{itemize}
		\item Průmysl
		\begin{itemize}
			\item Kontrola povrchových vad (praskliny, nečistoty).
			\item Inspekce mikrostruktur polovodičů, čipů.Analýza opotřebení a korozních procesů.
		\end{itemize}
		\item Elektronika
		\begin{itemize}
			\item Inspekce mikrostruktur polovodičů, čipů.
			\item Analýza poruch v integrovaných obvodech.
		\end{itemize}
		\item Forenzní vědy
		\begin{itemize}
			\item Vyšetřování stop (vlákna, zbytky střelného prachu).
			\item Analýza povrchů při kriminalistických expertízách.
		\end{itemize}
	\end{itemize}
	\vfill
}

\frame{
	\frametitle{}
	\vfill
	\textbf{SEM $\mu$-reaktor}
	\begin{itemize}
		\item Umožňuje \textit{in-situ} studium reakcí.
		\item Například růst nanovláken suboxidu wolframu, \ce{W18O49}.
	\end{itemize}
	\begin{figure}
		\adjincludegraphics[height=.35\textheight]{img/SEM-reactor.jpg}
		\caption*{Pěstování nanovláken z oxidů wolframu.\footnote[frame]{Zdroj: \href{https://doi.org/10.1021/acs.cgd.3c01094}{\ce{W18O49} Nanowhiskers Decorating \ce{SiO2 } Nanofibers: Lessons from \textit{In Situ} SEM/TEM Growth to Large Scale Synthesis and Fundamental Structural Understanding}}}
	\end{figure}
	\vfill
}

\frame{
	\frametitle{}
	\vfill
	\begin{figure}
		\adjincludegraphics[height=.67\textheight]{img/Germanium_Telluride_nanowires.jpg}
		\caption*{Nanovlákna GeTe.\footnote[frame]{Zdroj: \href{https://commons.wikimedia.org/wiki/File:Germanium_Telluride_nanowires.jpg}{Fionán/Commons}}}
	\end{figure}
	\vfill
}

\frame{
	\frametitle{}
	\vfill
	\begin{figure}
		\adjincludegraphics[height=.75\textheight]{img/Al2O3-SEM.png}
		\caption*{Porézní \ce{Al2O3}}
	\end{figure}
	\vfill
}

\frame{
	\frametitle{}
	\vfill
	\begin{figure}
		\adjincludegraphics[height=.75\textheight]{img/Al2O3-1.jpg}
		\caption*{Sférické částice \ce{Al2O3} (zvětšení 4 000$\times$)}
	\end{figure}
	\vfill
}

\frame{
	\frametitle{}
	\vfill
	\begin{figure}
		\adjincludegraphics[height=.75\textheight]{img/Al2O3-2.jpg}
		\caption*{Sférické částice \ce{Al2O3} (zvětšení 20 000$\times$)}
	\end{figure}
	\vfill
}

\frame{
	\frametitle{}
	\vfill
	\begin{figure}
		\adjincludegraphics[height=.67\textheight]{img/Quasimodopsis_riedeli_SEM.jpg}
		\caption*{SEM snímek \textit{Quasimodopsis riedeli}.\footnote[frame]{Zdroj: \href{https://commons.wikimedia.org/wiki/File:Quasimodopsis_riedeli_SEM.jpg}{Michael S. Caterino/Commons}}}
	\end{figure}
	\vfill
}

\subsection{Kryoelektronová mikroskopie}
\frame{
	\frametitle{}
	\vfill
	\begin{itemize}
		\item \textbf{Kryoelektronová mikroskopie} (cryo-EM) je metoda zobrazování biologických vzorků při velmi nízkých teplotách, která umožňuje studovat struktury biomolekul v jejich přirozeném stavu bez nutnosti krystalizace.
		\item Vzorek (např. protein, virus) se nanese na mřížku a okamžitě se zmrazí v kapalném ethanu při teplotě blízké -196 °C. Tím se vytvoří amorfní led, který zabraňuje tvorbě krystalů vody a uchovává molekuly v nativní konformaci.
		\item Vzorek se vloží do elektronového mikroskopu vybaveného kryogenním systémem. Elektronový svazek prochází vzorkem a vytváří snímky s vysokým rozlišením.
		\item Získají se tisíce dvourozměrných projekcí jednotlivých částic v různých orientacích.
		\item Pomocí algoritmů se tyto snímky zarovnají a složí do trojrozměrného modelu s rozlišením až na úroveň atomů.
	\end{itemize}
	\vfill
}

\frame{
	\frametitle{}
	\vfill
	\begin{columns}
		\begin{column}{.5\textwidth}
			\begin{figure}
				\adjincludegraphics[height=.63\textheight]{img/Titan_Krios_University_of_Leeds.jpg}
				\caption*{cryo-EM mikroskop Titan Krios.\footnote[frame]{Zdroj: \href{https://commons.wikimedia.org/wiki/File:Titan_Krios_University_of_Leeds.jpg}{Hiramano92/Commons}}}
			\end{figure}
		\end{column}
		\begin{column}{.5\textwidth}
			\begin{figure}
				\adjincludegraphics[height=.63\textheight]{img/CroV_TEM.jpg}
				\caption*{Snímky virů z cryo-EM mikroskopu, úsečka odpovídá vzdálenosti 200 nm.\footnote[frame]{Zdroj: \href{https://doi.org/10.1038/s41598-017-05824-w}{Nature}}}
			\end{figure}
		\end{column}
	\end{columns}
	\vfill
}

\subsection{Mikroskopie atomárních sil}
\frame{
	\frametitle{}
	\vfill
	\begin{itemize}
		\item \textbf{Mikroskopie atomárních sil} (AFM -- \textit{Atomic Force Microscopy}) sleduje interakce mezi velmi ostrým hrotem a povrchem vzorku.
		\item Hrot se přibližuje k povrchu vzorku.
		\item Mezi hrotem a povrchem působí mezimolekulární síly (van der Waalsovy, elektrostatické, kapilární).
		\item Tyto síly způsobují ohyb cantileveru, který je snímán laserem.
		\item Z ohybu se vypočítá velikost síly pomocí Hookova zákona:
		\begin{itemize}
			\item $F = k \cdot \Delta z$
		\end{itemize}
		\item Rozlišujeme tři měřící režimy:
		\begin{enumerate}
			\item Kontaktní (Contact mode) -- hrot je v přímém kontaktu s povrchem.
			\item Tapping (Intermittent contact) -- hrot kmitá a jen se dotýká povrchu.
			\item Bezkontaktní -- hrot se pohybuje nad povrchem a měří slabé přitažlivé síly.
		\end{enumerate}
		\item Získáme topografickou mapu povrchu s rozlišením až na úrovni jednotlivých atomů. Dále získáme i informace o mechanických, elektrických či magnetických vlastnostech povrchu vzorku.
	\end{itemize}
	\vfill
}

\frame{
	\frametitle{}
	\vfill
	\begin{columns}
		\begin{column}{.4\textwidth}
			\begin{figure}
				\adjincludegraphics[width=\textwidth]{img/Sony_PS-T15.jpg}
				\caption*{Gramofon Sony.\footnote[frame]{Zdroj: \href{https://commons.wikimedia.org/wiki/File:Sony_PS-T15_(9374966005).jpg}{Jacques/Commons}}}
			\end{figure}
		\end{column}
		\begin{column}{.6\textwidth}
			\begin{figure}
				\adjincludegraphics[width=.95\textwidth]{img/Cartridge-stylus-mmc2.jpg}
				\caption*{Snímací hlava gramofonu.\footnote[frame]{Zdroj: \href{https://commons.wikimedia.org/wiki/File:Cartridge-stylus-mmc2.jpg}{Heje/Commons}}}
			\end{figure}
		\end{column}
	\end{columns}
	\vfill
}

\frame{
	\frametitle{}
	\vfill
	\begin{figure}
		\adjincludegraphics[height=.67\textheight]{img/AFMsetup.jpg}
		\caption*{Schéma AFM.\footnote[frame]{Zdroj: \href{https://commons.wikimedia.org/wiki/File:AFMsetup.jpg}{KristianMolhave/Commons}}}
	\end{figure}
	\vfill
}

\frame{
	\frametitle{}
	\vfill
	\begin{figure}
		\adjincludegraphics[height=.67\textheight]{img/AFM_-_detail.jpg}
		\caption*{AFM mikroskop.\footnote[frame]{Zdroj: \href{https://commons.wikimedia.org/wiki/File:AFM_-_detail.jpg}{Laundry/Commons}}}
	\end{figure}
	\vfill
}

\frame{
	\frametitle{}
	\vfill
	\begin{figure}
		\adjincludegraphics[height=.67\textheight]{img/AFM_3D_Topography.jpg}
		\caption*{AFM snímek nanočástic palladia.\footnote[frame]{Zdroj: \href{https://commons.wikimedia.org/wiki/File:AFM_3D_Topography_Image_of_Palladium_Nanoparticles_on_Chitosan_Film.tif}{Mehrabanian/Commons}}}
	\end{figure}
	\vfill
}

\frame{
	\frametitle{}
	\vfill
	\textbf{AFM-FTIR}
	\begin{itemize}
		\item AFM lze kombinovat s FTIR spektroskopií a sledovat tak distribuci vybraných funkčních skupin na povrchu vzorku.
		\item To lze využít pro studium polymerů, proteinlů, buněk, palivových článků, MOFů i polovodičů.
	\end{itemize}
	\begin{figure}
		\adjincludegraphics[height=.35\textheight]{img/AFM-FTIR-photograph.jpg}
		\caption*{AFM-FTIR spektrometr.\footnote[frame]{Zdroj: \href{https://commons.wikimedia.org/wiki/File:AFM-FTIR-photograph.jpg}{Catsmeat/Commons}}}
	\end{figure}
	\vfill
}

\frame{
	\frametitle{}
	\vfill
	\begin{figure}
		\adjincludegraphics[height=.67\textheight]{img/AFM-IR_of_Streptomyces_bacteria.png}
		\caption*{AFM-IR mapa proteinů a triglyceridů v bakterii. Uprostřed absorbance amidových skupin a vpravo absorbance karbonylů.\footnote[frame]{Zdroj: \href{https://commons.wikimedia.org/wiki/File:AFM-IR_of_Streptomyces_bacteria.png}{Cbprater/Commons}}}
	\end{figure}
	\vfill
}

\section{Elektronová mikroskopie a Brno}
\frame{
	\frametitle{}
	\vfill
	\begin{columns}
		\begin{column}{.5\textwidth}
			\begin{figure}
				\adjincludegraphics[height=.7\textheight]{img/Brno_Montage_IV.png}
				\caption*{Brno.\footnote[frame]{Zdroj: \href{https://commons.wikimedia.org/wiki/File:Brno_Montage_IV.png}{Commons}}}
			\end{figure}
		\end{column}
		\begin{column}{.5\textwidth}
			\begin{figure}
				\adjincludegraphics[height=.7\textheight]{img/Titan_Krios_University_of_Leeds.jpg}
				\caption*{Titan Krios Cryo-EM.\footnote[frame]{Zdroj: \href{https://commons.wikimedia.org/wiki/File:Titan_Krios_University_of_Leeds.jpg}{Hiramano92/Commons}}}
			\end{figure}
		\end{column}
	\end{columns}
	\vfill
}

\frame{
	\frametitle{}
	\vfill
	\begin{itemize}
		\item Zakladatelem elektronové mikroskopie v Československu byl Armin Delong, který působil v Brně.
		\item V současnosti (2025) v Brně vzniká třetina všech elektronových mikroskopů na světě, jde asi o 700 kusů.
		\item Kromě toho zde probíhá i vývoj nových přístrojů a samozřejmě i klasický výzkum využívající elektronovou mikroskopii.
		\item V květnu 2025 bylo v Brně otevřeno Centrum elektronové mikroskopie, budova s velmi unikátní technologií, ve které jsou k dispozici dva transmisní a tři skenovací elektronové mikroskopy.\footnote[frame]{\href{https://www.avcr.cz/cs/o-nas/aktuality/Nove-Centrum-elektronove-mikroskopie-v-Brne-otevira-dvere-svetove-vede/}{Nové Centrum elektronové mikroskopie v Brně otevírá dveře světové vědě}}
		\item Od roku 2017 se v Brně konají \textit{Dny elektronové mikroskopie}.\footnote[frame]{\href{https://dembrno.cz/}{Dny elektronové mikroskopie}}
	\end{itemize}
	\vfill
}

\frame{
	\frametitle{}
	\vfill
	\begin{figure}
		\adjincludegraphics[height=.75\textheight]{img/Otevreni-Centra-elektronove-mikroskopie_UFM.jpg}
		\caption*{Centrum elektronové mikroskopie v Brně.\footnote[frame]{Zdroj: \href{https://www.avcr.cz/cs/o-nas/aktuality/Nove-Centrum-elektronove-mikroskopie-v-Brne-otevira-dvere-svetove-vede/}{AVČR}}}
	\end{figure}
	\vfill
}

\subsection{Armin Delong}
\frame{
	\frametitle{}
	\vfill
	\begin{columns}
		\begin{column}{.7\textwidth}
			\begin{itemize}
				\item 29. ledna 1925 Ostrava -- 5. října 2017 Brno
				\item Český vědec, fyzik a zakladatel elektronové mikroskopie v Československu.\footnote[frame]{\href{https://www.novinky.cz/clanek/veda-skoly-delong-by-oslavil-stovku-z-brna-je-diky-nemu-svetove-centrum-elektronove-mikroskopie-40506795}{Delong by oslavil stovku. Z Brna je díky němu světové centrum elektronové mikroskopie}}
				\item Působil na MU a VUT.
				\item Byl spoluautorem stolního transmisního elektronového mikroskopu, který vyráběla firma Tesla. Prodalo se ho celkem kolem 1100 kusů a byl oceněn zlatou medailí na výstavě EXPO58 v Bruselu v roce 1958.
				\item Podílel se na založení \textit{Ústavu přístrojové techniky ČSAV}, v jehož čele stál více než 30 let.
				\item Mezi lety 1967 až 1971 byl vedoucím Katedry fyziky pevné fáze na Přírodovědecké fakultě Masarykovy univerzity.
			\end{itemize}
		\end{column}
		\begin{column}{.35\textwidth}
			\begin{figure}
				\adjincludegraphics[width=\textwidth]{img/Delong_2014.jpg}
				\caption*{Armin Delong v roce 2014.\footnote[frame]{Zdroj: \href{https://commons.wikimedia.org/wiki/File:Delong_2014.jpg}{OISV/Commons}}}
			\end{figure}
		\end{column}
	\end{columns}
	\vfill
}

\frame{
	\frametitle{}
	\vfill
	\begin{figure}
		\adjincludegraphics[height=.67\textheight]{img/Electron_microscope_Mamut.jpg}
		\caption*{Elektronový mikroskop Mamut v prostorách UPM.\footnote[frame]{Zdroj: \href{https://commons.wikimedia.org/wiki/File:Electron_microscope_Mamut_at_the_Institute_of_Scientific_Instruments_of_the_Czech_Academy_of_Science_in_Brno_(1).jpg}{AVČR/Commons}}}
	\end{figure}
	\vfill
}

\subsection{Výroba mikroskopů v Brně}
\frame{
	\frametitle{}
	\vfill
	\begin{itemize}
		\item V Brně je vyrobena zhruba třetina všech elektronových mikroskopů na světě.
		\item Jsou za to odpovědné tři firmy:
		\begin{itemize}
			\item Delong Instruments
			\item Tescan
			\item Thermo Fisher Scientific
		\end{itemize}
		\item Tržby v tomto odvětví neustále vzrůstají:
	\end{itemize}

	\begin{center}
		\begin{tabular}{|l|r@{,}l|}
		\hline
		\textbf{Rok} & \multicolumn{2}{c|}{\textbf{Tržby [mld USD]}} \\\hline
			2019 & 2 & 65 \\\hline
			2020 & 2 & 6 \\\hline
			2021 & 3 & 1 \\\hline
			2022 & 4 & 5 \\\hline
			2023 & 4 & 2 \\\hline
			2024 & \~5 & 1 \\\hline
		\end{tabular}
	\end{center}
	\vfill
}

\frame{
	\frametitle{}
	\vfill
	\begin{columns}
		\begin{column}{.6\textwidth}
			\textbf{Delong Instruments}
			\begin{itemize}
				\item Založena v roce 1992 v Brně.
				\item Sídlí v Králově poli a od roku 2018 má výrobní závod v Boskovicích.
				\item Vyrábí nízkonapěťové TEM mikroskopy LVEM a elektronové zdroje DIGUN.
			\end{itemize}
		\end{column}
		\begin{column}{.45\textwidth}
			\begin{figure}
				\adjincludegraphics[width=\textwidth]{img/Delong_Instruments.png}
			\end{figure}
		\end{column}
	\end{columns}
	\vfill
}

\frame{
	\frametitle{}
	\vfill
	\begin{columns}
		\begin{column}{.6\textwidth}
			\textbf{Tescan Group}
			\begin{itemize}
				\item Založena v roce 1991 jako firma Tescan pracovníky česká formy Tesla.
				\item Sídlí v Kohoutovicích.
				\item Vyrábí SEM mikroskopy pro široké spektrum aplikací, včetně biologie a životního prostředí.
				\item Zaměstnává více než 700 lidí v Česku.
				\item V současnosti působí po celém světě.
			\end{itemize}
		\end{column}
		\begin{column}{.45\textwidth}
			\begin{figure}
				\adjincludegraphics[width=\textwidth]{img/Tescan.jpg}
				\caption*{Sídlo firmy Tescan v Kohoutovicích.}
			\end{figure}
		\end{column}
	\end{columns}
	\vfill
}

\frame{
	\frametitle{}
	\vfill
	\begin{columns}
		\begin{column}{.6\textwidth}
			\textbf{Thermo Fisher Scientific}
			\begin{itemize}
				\item V roce 1993 založena v Jundrově jako firma Delmi, mezi zakladateli byl i Petr Střelec.\footnote[frame]{\href{https://thermofisher.jobs.cz/blog/jak-se-z-party-nadsencu-stala-jedna-z-nejvetsich-firem-na-jizni-morave-brnenska-pobocka-thermo-fisher-scientific-si-pripomela-30-vyroci}{Jak se z party nadšenců stala jedna z největších firem na jižní Moravě.}}
				\item Později přejmenována na FEI Company.
				\item V roce 2016 došlo ke spojení FEI a Thermo Fisher, firma vyrábí elektronové mikroskopy a spektrometry. Zaměstnává okolo 1800 lidí, tržby za rok 2022 činili skoro 20 mld Kč.
				\item V roce 2025 začala stavba nové haly v areálu Královopolských strojíren, po dokončení zde bude pracovat 500 zaměstnanců.\footnote[frame]{\href{https://www.seznamzpravy.cz/clanek/ekonomika-firmy-americky-vyrobce-mikroskopu-investuje-v-brne-miliardy-278145}{Americký výrobce mikroskopů investuje v Brně miliardy}}
			\end{itemize}
		\end{column}
		\begin{column}{.45\textwidth}
			\begin{figure}
				\adjincludegraphics[width=\textwidth]{img/Thermo_Fisher_Scientific_logo.png}
			\end{figure}
		\end{column}
	\end{columns}
	\vfill
}

\section{Závěr}
\frame{
	\vfill
	\centering \Huge
	\textbf{Děkuji za pozornost} \\[2ex]

	\large
	Zdeněk Moravec\\
	\href{https://is.muni.cz/www/moravec/brnoc/2025-zima/}{is.muni.cz/www/moravec/brnoc/2025-zima/}\\
	hugo@chemi.muni.cz
	\\
	\adjincludegraphics[height=.3\textheight]{img/qr.png}
	\vfill
}


\end{document}